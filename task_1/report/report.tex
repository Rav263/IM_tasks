\documentclass[a4peper, 12pt, titlepage, finall]{extreport}

%различные пакеты

\usepackage[T1, T2A]{fontenc}
\usepackage[russian]{babel}
\usepackage{tikz}
\usepackage{geometry}
\usepackage{indentfirst}
\usepackage{fontspec}


\usetikzlibrary{positioning, arrows}

\geometry{a4paper, left = 15mm, top = 10mm, bottom = 20mm, right = 15mm}

\setmainfont{Times New Roman}
\setmonofont{Courier New}
\setcounter{secnumdepth}{0}


\begin{document}

    \begin{titlepage}
        \begin{center}
            {\small \sc Московский государственный университет имени М.~В.~Ломоносова\\
            Факультет вычислительной математики и кибирнетики\\
            Кафедра автоматизации систем вычислительных комплексов\\}
            \vfill
            {\large \sc Отчёт по заданию №1}\\~\\

            {\large \bf Оптимизация программ.}\\~\\

            {\large \bf Вариат №2}
        \end{center}
        
        \begin{flushright}
            \vfill
            {Никифоров Никита Игоревич, 321 группа}
        \end{flushright}

        \begin{center}
            \vfill
            {\small Москва\\2019}
        \end{center}
    \end{titlepage}

    \tableofcontents
    \newpage

    \section{Постановка задачи}
        В данной работе необходимо было оптимизировать данную программу. Изначальная программа была реализованная на языке си, для её оптимизации возможно было использовать всё, при условии что сохраняется семантика программы.

    \newpage
    \section{Описание методов оптимизации}
        Изначально было принято решение попробовать скомпилировать программу с максимальной оптимизацией компилятора {\ttfamily -03}, что дало прирост примерно в 4.5 раза. 
        После было прочитанно несколько статей на тему оптимизации вычислений с матрицами. 
        Практически в каждой предлогалось использовать векторные инструкции процессора. 
        После просмотра ассемблерного кода, компилятора, выяснилось что после применения {\ttfamily -O3}, компилятор сам применил векторные инструкции (Это видно на данном куске {\ttfamily objdump -D}). \\
    


    {
        \setlength{\parindent}{0em}

        {
            \setlength{\leftskip}{3em} \ttfamily
            11a4:	48 83 c0 08          	add    0x8, \%rax \\
            11a8:	c4 a1 7b 10 04 06    	vmovsd (rsi,r8,1),\%xmm0\\
            11ae:	c5 fb 58 40 f8       	vaddsd -0x8(\%rax)\,\%xmm0,\%xmm0\\
            11b3:	c5 f3 58 c8          	vaddsd \%xmm0,\%xmm1,\%xmm1\\
            11b7:	c5 fb 11 09          	vmovsd \%xmm1,(\%rcx) \\
    
        }
    }
    Соответственно необходимоть использовать ихвручную отпала.


\end{document}
