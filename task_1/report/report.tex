\documentclass[a4peper, 12pt, titlepage, finall]{extreport}

%различные пакеты

\usepackage[T1, T2A]{fontenc}
\usepackage[russian]{babel}
\usepackage{tikz}
\usepackage{geometry}
\usepackage{indentfirst}
\usepackage{fontspec}


\usetikzlibrary{positioning, arrows}

\geometry{a4paper, left = 15mm, top = 10mm, bottom = 20mm, right = 15mm}

\setmainfont{Times New Roman}
\setmonofont{Courier New}
\setcounter{secnumdepth}{0}


\begin{document}

    \begin{titlepage}
        \begin{center}
            {\small \sc Московский государственный университет имени М.~В.~Ломоносова\\
            Факультет вычислительной математики и кибирнетики\\
            Кафедра автоматизации систем вычислительных комплексов\\}
            \vfill
            {\large \sc Отчёт по заданию №1}\\~\\

            {\large \bf Оптимизация программ.}\\~\\

            {\large \bf Вариат №2}
        \end{center}
        
        \begin{flushright}
            \vfill
            {Никифоров Никита Игоревич, 321 группа}
        \end{flushright}

        \begin{center}
            \vfill
            {\small Москва\\2019}
        \end{center}
    \end{titlepage}

    \tableofcontents
    \newpage

    \section{Постановка задачи}
        Основная задача в этой работе -- оптимизировать данную программу. Изначальная программа была реализованная на языке си, для её оптимизации возможно было использовать всё, при условии что сохраняется семантика программы.

    \section{Конфигурация ВС}
        \begin{tabular}{lc}
            Процессор & AMD Ryzen 7 2700X 4.2GHz\\
            Видеокарта & Nvidia GTX 1060 3GB\\
            Оперативная память & DDR4 16 Gb 3400 MHz\\
            Операционная система & Arch Linux \\
        \end{tabular}

    \section{История исследования}
        Изначально было принято решение попробовать скомпилировать программу с максимальной оптимизацией компилятора {\ttfamily -03}, что дало прирост примерно в 4.5 раза. 
        После было прочитанно несколько статей на тему оптимизации вычислений с матрицами. 
        Практически в каждой предлогалось использовать векторные инструкции процессора. 
        После просмотра ассемблерного кода, компилятора, выяснилось что после применения {\ttfamily -O3}, компилятор сам применил векторные инструкции (Это видно на данном куске {\ttfamily objdump -D}). \\
    


        {
            \setlength{\parindent}{0em}

            {
                \setlength{\leftskip}{3em} \ttfamily
                11a4:	48 83 c0 08          	add    0x8, \%rax \\
                11a8:	c4 a1 7b 10 04 06    	vmovsd (rsi,r8,1),\%xmm0\\
                11ae:	c5 fb 58 40 f8       	vaddsd -0x8(\%rax)\,\%xmm0,\%xmm0\\
                11b3:	c5 f3 58 c8          	vaddsd \%xmm0,\%xmm1,\%xmm1\\
                11b7:	c5 fb 11 09          	vmovsd \%xmm1,(\%rcx) \\
    
            }
        }
        Соответственно необходимоть использовать ихвручную отпала. После было принято решение распараллеливания данной задачи, поскольку в пользовании
        находится 8 ядерный 16 поточный процессор, в программе можно было использовать большое количество потоков (точнее все 16). 
        К счастью, задача очень просто раскладывалась на потоки, которые вообще не пересекались между собой. 
        После распараллеливания программы прирост составил ещё 7.5 раза. Как итог программа стала выполняться за 90мс, против изначальных 3200мс.
        Далее было принято решение попробовать провести вычисления на видеокарте. Была написана программа с использованием CUDA, 
        хотя и вычисления проходили быстро, технические функции (инициализация ГПУ, 
        копирование данных в память и синхронизация с основным потоком программы) занимали примерно 300мс, что уже было в три раза
        медленнее, чем вычисления на процессоре. Но при больших объёмах данных вычисления
        на ГПУ будут куда более быстрые, чем на процессоре.
        Спустя несколько дней, когда уже часть отчёта была написана, в голову пришла светлая
        идея, а что если реализовать данные вычисления не с помощью кубического алгоритма (как
        это было изначально), а придумать квадратичный алгоритм. И такой алгоритм был найден
        (подробное описание в параграфе ...), время выполнения на одном потоке стало 11мс, что
        дало общий прирост в 290 раз. К сожалению дальнейшие попытки распараллеливания
        квадратичного алгоритма не увенчались успехом (программа выполнялась либо больше,
        либо столько же)

    \newpage
    \section{Измерения времени}
        Все измерения производились с помощью утилиты {\ttfamily time}, по 5 раз на тест, и бралось среднее значение.\\
            \begin{tabular}{lccc}
                \bf описание & \bf время (mc) & \bf ускорение (раз) &\bf название файла \\
                Исходный код без оптимизаций & 3200 & 1 &p2var.c \\
                Исходный код скомпилированный с {\ttfamily -O3} & 2.6 & 1190 & test\_1.c \\
                Распараллеливание на 16 потоков & 500 & 6.4 & test\_4.c \\
                Распараллеливание на 16 потоков с {\ttfamily -O3} & 90 & 35.5 & test\_4.c \\
                Решение задачи на видеокарете спомощью {\ttfamily CUDA} & 500 & 6.4 & test.cu \\
                Решение задачи квадратичным алгоритмом & 11 & 290.9 & test\_6.c \\
            \end{tabular}
       
    \section{Oписание методов оптимизации}
        Применение оптимизаций компилятора — компилятор сам анализирует программу и
        оптимизирует её под конкретную архитектуру. \\
        Использование векторных инструкций процессора — позволяет процессору складывать
        вектора чисел за 1 одну инструкцию, что даёт огромное ускорение\\
        Использование ГПУ для вычислений — позволяет значительно ускорить вычисления с
        матрицами, для достаточно больших объёмов данных.\\
        Написание более оптимального алгоритма — если существует возможность реализовать
        алгоритм с меньшей асимптотической сложностью (а потом к нему применять предыдущие
        оптимизации) \\

        \subsection{Описание квадратичного алгоритма}
        Исходный алгоритм: дано две матрицы исходная {\ttfamily b[i][j]} и выходная {\ttfamily a[i][j]} , соответственно
        выходная матрица задавалась формулой {\ttfamily a[i][j] = b[i][k] + b[j][k]} для любых {\ttfamily i,j,k = 0....size.}
        Данная формула по смыслу означает что каждый элемент выходной матрицы равен сумме
        строк исходной матрицы с номера равными координатам данного элемента.
        Квадратичный алгоритм: можно создать дополнительный массив, в который записать суммы
        элементов по строкам исходной матрицы, и в двойном цикле просто посчитать элементы
        выходной матрицы. Т.е. {\ttfamily c[i] = b[i][0] + ... + b[i][size], a[i][j] = c[i] + c[j];} Тем самым
        асимптотическая сложность нового алгоритма — {\ttfamily $O(n^{2})$ }

\end{document}
